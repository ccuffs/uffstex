% opção global de idioma "brazilian"
% (mesmo que uffstex ou abntex2 façam algo do tipo)
% para escrever em inglês, colocar "english"
% (o último idioma é o padrão)
% esta opção é informada aos pacotes importados
% com \usepackage (por exemplo, cleveref ao
% nomear os links das referências) e para as
% regras de hifenização
\documentclass[serif, brazilian]{uffstex}
\usepackage[utf8]{inputenc}
\usepackage[T1]{fontenc}
\usepackage[
    style=abnt,
    sorting=nty % ordenar por nome, título e ano
    % a NBR 6023 não especifica desempate de ordenação...
]{biblatex}

\usepackage{booktabs}
\usepackage{csquotes}
% o abnTeX2 e a uffstex não obrigam
% a utilização de alguma fonte, então, se
% nenhuma for especificada, a padrão do
% LaTeX é utilizada (Computer Modern)
% as fontes do pacote newtx são recomendadas
% para quem deseja utilizar Times New Roman
\usepackage{newtxtext}
\usepackage{newtxmath}
% cleveref deve ser carregado após hyperref
% (carregado automaticamente pelo abntex2)
% e amsmath (pacote newtxmath aparentemente
% carrega amsmath)
\usepackage{cleveref}

\addbibresource{refs.bib}

\autor{CC UFFS}
\titulo{\uffstex}
\subtitulo{Classe de trabalhos acadêmicos da UFFS}
\curso{Ciência da Computação}
\local{Chapecó}
\data{\hoje}
\orientador{Fulano}
\coorientador{Beltrano}

\begin{document}
% ainda precisamos selecionar o idioma aqui
% para imprimir os rótulos no idioma correto,
% caso não seja português do brasil
% \selectlanguage{english}

% início da parte pré-textual
% (opcional, abnTeX2 faz automaticamente)
\pretextual

\imprimircapa

\imprimirfolhaderosto

% coloca o sumário no "sumário do PDF"
% (chamado de bookmark do PDF)
\pdfbookmark{\contentsname}{toc}
% versão com * não coloca o próprio sumário no sumário
\tableofcontents*
\clearpage

% início da parte textual
\textual

\chapter{Seção primária}

Conteúdo da seção primária.

\section{Seção secundária}

Conteúdo da seção secundária.

\subsection{Seção terciária}

Conteúdo da seção terciária.

\subsubsection{Seção quaternária}

Conteúdo da seção quaternária.

\subsubsubsection{Seção quinária}

Conteúdo da seção quinária.

\chapter{Citações}

Segundo \textcite{halim3rd}, codar um array de sufixos é mais fácil que uma árvore de sufixos.

Codar um array de sufixos é mais fácil que uma árvore de sufixos \cite{halim3rd}.

\chapter{Imagens}

Na \autoref{fig:logo_uffs} podemos ver a logo da \imprimirinstituicao.

Na \cref{fig:logo_uffs} podemos ver a logo da \imprimirinstituicao.

\begin{figure}[h]
  \centering
  \caption{Logo da UFFS}
  \includegraphics{logo.png}
  \legend{Fonte: UFFS}
  \label{fig:logo_uffs}
\end{figure}

\chapter{Alíneas}

A ABNT NBR 6024:2012 define que o texto que antecede as alíneas deve terminar com dois pontos; cada alínea deve iniciar com letra minúscula e terminar com ponto e vírgula, exceto a última, finalizada com ponto final, e a alínea que anteceder uma subalínea, finalizada com dois pontos:

\begin{alineas}
  \item item 1;
  \item item 2;
  \item item 3:
  \begin{alineas} % ou \begin{subalineas} ou \begin{incisos}
    \item item 3.1;
    \item item 3.2;
  \end{alineas}
  \item item 4.
\end{alineas}

Note que se a última alínea for uma subalínea, ela será finalizada com ponto final. Além disso, a ABNT define o ponto e vírgula para terminar uma subalínea, enquanto a normalização de trabalhos da UFFS determina o uso da vírgula neste caso. Como desconhecemos se a ABNT define alguma flexibilidade na customização desta regra, deixamos exemplificado o uso da regra original.

\chapter{Tabelas}

Exemplos de tabelas\footnote{Todas as seções devem conter algum texto entre elas, segundo o manual de normalização de trabalhos da UFFS.}.

\section{Tabelas comuns}

% tabelas comuns do LaTeX, com um pouco
% de beleza graças ao pacote booktabs
\begin{table}[h]
  \centering
  % \caption é o rótulo (título, identificação), que deve estar
  % acima do elemento (tabelas, figuras, etc)
  \caption{Países, capitais e população total aproximada}
  % @{} remove o espaço nas laterais
  \begin{tabular}{@{}llr@{}}
    \toprule % linha superior
    % notas de rodapé não se dão bem com tabelas, então
    % precisa usar \footnotemark com \footnotetext ao
    % invés de apenas \footnote{<texto>}
    País   & Capital  & População\footnotemark\\
    \midrule % linha do meio
    Brasil & Brasília & 208 milhões\\
    Itália & Roma     & 61 milhões\\
    Canadá & Ottawa   & 37 milhões\\
    China  & Pequim   & 1.4 bilhões\\
    \bottomrule % linha inferior
  \end{tabular}
  % \legend é a legenda para indicar a fonte e outras informações,
  % se necessário, e deve estar abaixo do elemento
  \legend{Fonte: elaborado pelos autores com dados do Google}
  % etiqueta para referenciar o elemento no texto
  \label{tab:paises}
\end{table}

\section{Tabelas do padrão IBGE}

\begin{table}[htb]
  % \ibgetab{<rótulo e etiqueta>}{<conteúdo>}{<legenda(s)>}
  \ibgetab{%
    \caption{Um Exemplo de tabela conforme o padrão IBGE}
    \label{tabela-ibge}
  }{
    \begin{tabular}{llr}
      \toprule
      País   & Capital  & População\\
      \midrule \midrule
      Brasil & Brasília & 208 milhões\\
      Itália & Roma     & 61 milhões\\
      Canadá & Ottawa   & 37 milhões\\
      China  & Pequim   & 1.4 bilhões\\
      \bottomrule
    \end{tabular}
  }{
    \fonte{elaborado pelos autores, com dados do Google e comandos retirados do manual do \abnTeX}
    \nota{Uma nota.}
    \nota[Anotações]{Outra nota, mas com prefixo diferente.}
  }
\end{table}

\footnotetext{População do país, não da capital!}

\postextual

\printbibliography

\end{document}
