% Opção global de idioma "brazilian"
% (mesmo que uffstex ou abntex2 façam algo do tipo).
% Para escrever em inglês, colocar "english"
% (o último idioma é o padrão).
% Esta opção é informada aos pacotes importados
% com \usepackage (por exemplo, cleveref ao
% nomear os links das referências) e para as
% regras de hifenização.
% A opção "serif" estiliza os títulos com a versão
% serifada da fonte escolhida (abntex2 utiliza
% estilo sem serifa por padrão).
\documentclass[serif, brazilian]{uffstex}
\usepackage[utf8]{inputenc}
\usepackage[T1]{fontenc}
\usepackage[
  % Estilo numérico é preferível e é
  % permitido pela ABNT e pela UFFS :)
  style=abnt-numeric,
  % Para o estilo autor-data:
  % style=abnt
  % Ordenação:
  % sorting=nty % por nome, título e ano
  % (a NBR 6023 não especifica desempate de ordenação...)
]{biblatex}

\usepackage{xcolor}
\usepackage{booktabs}
\usepackage{csquotes}
% O abnTeX2 e o uffstex não obrigam
% a utilização de alguma fonte, então, se
% nenhuma for especificada, a padrão do
% LaTeX é utilizada (Computer Modern).
% As fontes do pacote newtx são recomendadas
% para quem deseja utilizar Times New Roman.
\usepackage{newtxtext}
\usepackage{newtxmath}
% cleveref deve ser carregado após hyperref
% (que é carregado automaticamente pelo abntex2)
% e amsmath (pacote newtxmath aparentemente
% carrega amsmath).
\usepackage{cleveref}

% verde padrão da UFFS segundo a identidade visual
\providecolor{uffsgreen}{RGB}{0, 105, 62}

% Adiciona arquivo de referências para o biblatex
\addbibresource{refs.bib}

% Comandos de dados
\autor{CC UFFS}
\titulo{\uffstex}
\subtitulo{Classe de trabalhos acadêmicos da UFFS}
\curso{Ciência da Computação}
\local{Chapecó}
\data{\hoje}
\orientador{Fulano}
\coorientador{Beltrano}

\begin{document}
% Ainda precisamos selecionar o idioma aqui
% para imprimir os rótulos no idioma correto,
% caso não seja português do Brasil.
% \selectlanguage{english}

% Início da parte pré-textual
% (opcional, abnTeX2 faz automaticamente).
\pretextual

\imprimircapa

\imprimirfolhaderosto

\pdfbookmark{\listfigurename}{lof}
\listoffigures*
\cleardoublepage%

% Coloca o sumário no "sumário do PDF"
% (chamado de bookmark do PDF).
\pdfbookmark{\contentsname}{toc}
% Versão com * não coloca o próprio sumário no sumário
\tableofcontents*
\cleardoublepage

% Início da parte textual
\textual

\chapter{Seção primária}

Conteúdo da seção primária.

\section{Seção secundária}

Conteúdo da seção secundária.

\subsection{Seção terciária}

Conteúdo da seção terciária.

\subsubsection{Seção quaternária}

Conteúdo da seção quaternária.

\subsubsubsection{Seção quinária}

Conteúdo da seção quinária.

\chapter{Citações}

\section{Indiretas}

% Para citar com número de página, basta
% informá-la como parâmetro opcional.
Segundo \textcite{halim3rd}, codar um array de sufixos é mais fácil que uma árvore de sufixos. Ou então: codar um array de sufixos é mais fácil que uma árvore de sufixos~\cite[253]{halim3rd}.

\section{Diretas}

Um dos trechos mais famosos do livro O Pequeno Príncipe é \enquote{O essencial é invisível aos olhos}~\cite[58]{exupery15}.

\subsection{Com mais de três linhas}

\begin{citacao}
\textit{A Léon Werth.}

Peço perdão às crianças por ter dedicado este livro a um adulto. Tenho uma boa desculpa: esse adulto é o meu melhor amigo no mundo. Tenho outra desculpa: esse adulto pode entender tudo, até livros para crianças. Tenho uma terceira desculpa: esse adulto mora na França, onde sente fome e frio. Também precisa ser consolado. Se todas essas desculpas não bastarem, então quero dedicar este livro à criança que esse adulto foi um dia. Todos os adultos primeiro foram crianças. (Mas poucos se lembram disso.) Portanto, corrijo minha dedicatória:

\textit{A Léon Werth quando era pequeno.}~\cite{exupery15}
\end{citacao}

\chapter{Alíneas}

A ABNT NBR 6024:2012 define que o texto que antecede as alíneas deve terminar com dois pontos; cada alínea deve iniciar com letra minúscula e terminar com ponto e vírgula, exceto a última, finalizada com ponto final, e a alínea que anteceder uma subalínea, finalizada com dois pontos:

\begin{alineas}
  \item item 1;
  \item item 2;
  \item item 3:
  \begin{alineas} % ou \begin{subalineas} ou \begin{incisos}
    \item item 3.1;
    \item item 3.2;
  \end{alineas}
  \item item 4.
\end{alineas}

Note que se a última alínea for uma subalínea, ela será finalizada com ponto final. Além disso, a ABNT define o ponto e vírgula para terminar uma subalínea, enquanto a normalização de trabalhos da UFFS determina o uso da vírgula neste caso. Como desconhecemos se a ABNT define alguma flexibilidade na customização desta regra, deixamos exemplificado o uso da regra original.

\chapter{Tabelas}

Exemplos de tabelas\footnote{Todas as seções devem conter algum texto entre elas, segundo o manual de normalização de trabalhos da UFFS.}.

\section{Tabelas comuns}

% Tabelas comuns do LaTeX, com um pouco
% de beleza graças ao pacote booktabs.
\begin{table}[h]
  \centering
  % \caption é o rótulo (título, identificação), que deve estar
  % acima do elemento (tabelas, figuras, etc).
  \caption{Países, capitais e população total aproximada}%
  \label{tab:paises}
  % @{} remove o espaço nas laterais.
  \begin{tabular}{@{}llr@{}}
    \toprule % linha superior
    % Notas de rodapé não se dão bem com tabelas, então
    % precisa usar \footnotemark com \footnotetext ao
    % invés de apenas \footnote{<texto>}.
    País   & Capital  & População\footnotemark\\
    \midrule % linha do meio
    Brasil & Brasília & 208 milhões\\
    Itália & Roma     & 61 milhões\\
    Canadá & Ottawa   & 37 milhões\\
    China  & Pequim   & 1.4 bilhões\\
    \bottomrule % linha inferior
  \end{tabular}
  \fonte{elaborado pelos autores com dados do Google}
  \nota{Outras informações, se necessário.}
  % Etiqueta para referenciar o elemento no texto.
\end{table}

\section{Tabelas do padrão IBGE}

\begin{table}[htb]
  % \ibgetab{<rótulo e etiqueta>}{<conteúdo>}{<legenda(s)>}
  \ibgetab{%
    \caption{Um Exemplo de tabela conforme o padrão IBGE}%
    \label{tabela-ibge}
  }{
    \begin{tabular}{llr}
      \toprule
      País   & Capital  & População\\
      \midrule \midrule
      Brasil & Brasília & 208 milhões\\
      Itália & Roma     & 61 milhões\\
      Canadá & Ottawa   & 37 milhões\\
      China  & Pequim   & 1.4 bilhões\\
      \bottomrule
    \end{tabular}
  }{
    \fonte{elaborado pelos autores, com dados do Google e comandos retirados do manual do \abnTeX}
    \nota{Uma nota.}
    \nota[Anotações]{Outra nota, mas com prefixo diferente.}
  }
\end{table}

\footnotetext{População do país, não da capital!}

\chapter{Imagens}

Na \autoref{fig:logo_uffs} podemos ver a nova assinatura visual da \imprimirinstituicao.

\begin{figure}[h]
  \centering
  \caption{Nova assinatura visual da UFFS}%
  \label{fig:logo_uffs}
  \includegraphics[scale=0.5]{logo.png}
  \fonte{UFFS}
\end{figure}

Na \cref{fig:logo_uffs_ibge} podemos ver a nova assinatura visual da \imprimirinstituicao\ reaproveitando o ambiente das tabelas do IBGE:

\begin{figure}[h]
  \ibgetab{
    \caption{Nova assinatura visual da UFFS}%
    \label{fig:logo_uffs_ibge}
  }{
    \includegraphics[scale=0.5]{logo.png}
  }{
    \fonte{UFFS}
  }
\end{figure}

\chapter{Miscelânea}

\textcolor{uffsgreen}{Podemos utilizar a cor verde oficial da UFFS através
do pacote \texttt{xcolor} e do comando:}

\begin{verbatim}
  \providecolor{uffsgreen}{RGB}{0, 105, 62}
\end{verbatim}

\postextual

% Para imprimir as referências,
% se utilizar o pacote biblatex (recomendado),
% como neste exemplo:
\printbibliography[heading=abnt]
% E se utilizar o abntex2cite, consultar a documentação:
% https://www.abntex.net.br/#manuais-do-abntex2
% ou
% https://ctan.org/pkg/abntex2
% (item "Ci­ta­tion sup­port pack­age doc­u­men­ta­tion")

\end{document}
